%
% exemplo genérico de uso da classe iiufrgs.cls
% $Id: iiufrgs.tex,v 1.1.1.1 2005/01/18 23:54:42 avila Exp $
%
% This is an example file and is hereby explicitly put in the
% public domain.
%
%% \documentclass[ppgc,mestrado]{iiufrgs}
% um tipo específico de monografia pode ser informado como parâmetro opcional:
%\documentclass[tese]{iiufrgs}
% monografias em inglês devem receber o parâmetro `english':
\documentclass[diss,english]{iiufrgs}
% a opção `openright' pode ser usada para forçar inícios de capítulos
% em páginas ímpares
% \documentclass[openright]{iiufrgs}
% para gerar uma versão somente-frente, basta utilizar a opção `oneside':
% \documentclass[oneside]{iiufrgs}
\usepackage[T1]{fontenc}        % pacote para conj. de caracteres correto
\usepackage[utf8]{inputenc}   % pacote para acentuação
\usepackage{graphicx}           % pacote para importar figuras
\usepackage{bibabnt}           % pacote para citacoes
\usepackage{times}              % pacote para usar fonte Adobe Times
%\usepackage{mathptmx}          % p/ usar fonte Adobe Times nas fórmulas

%
% Informações gerais
%
\title{A comparison of recommender systems for crowdfunding projects.}


\author{Carniel Benin}{Adriano}
% alguns documentos podem ter varios autores:
%\author{Flaumann}{Frida Gutenberg}
%\author{Flaumann}{Klaus Gutenberg}

% orientador e co-orientador são opcionais (não diga isso pra eles :))
\advisor[Prof.~Dr.]{Castro da Silva}{Bruno}
%\coadvisor[Prof.~Dr.]{Knuth}{Donald Ervin}

% a data deve ser a da defesa; se nao especificada, são gerados
% mes e ano correntes
%\date{maio}{2001}

% o nome do curso pode ser redefinido (ex. para TCs)
\course{ciencia da computação}

% O local de realização do trabalho pode ser especificado (ex. para TCs)
% com o comando \location:
\location{Porto Alegre}{RS}

% itens individuais da nominata podem ser redefinidos com os comandos
% abaixo:
% \renewcommand{\nominataReit}{Prof\textsuperscript{a}.~Wrana Maria Panizzi}
% \renewcommand{\nominataReitname}{Reitora}
% \renewcommand{\nominataPRE}{Prof.~Jos{\'e} Carlos Ferraz Hennemann}
% \renewcommand{\nominataPREname}{Pr{\'o}-Reitor de Ensino}
% \renewcommand{\nominataPRAPG}{Prof\textsuperscript{a}.~Joc{\'e}lia Grazia}
% \renewcommand{\nominataPRAPGname}{Pr{\'o}-Reitora Adjunta de P{\'o}s-Gradua{\c{c}}{\~a}o}
% \renewcommand{\nominataDir}{Prof.~Philippe Olivier Alexandre Navaux}
% \renewcommand{\nominataDirname}{Diretor do Instituto de Inform{\'a}tica}
% \renewcommand{\nominataCoord}{Prof.~Carlos Alberto Heuser}
% \renewcommand{\nominataCoordname}{Coordenador do PPGC}
% \renewcommand{\nominataBibchefe}{Beatriz Regina Bastos Haro}
% \renewcommand{\nominataBibchefename}{Bibliotec{\'a}ria-chefe do Instituto de Inform{\'a}tica}
% \renewcommand{\nominataChefeINA}{Prof.~Jos{\'e} Valdeni de Lima}
% \renewcommand{\nominataChefeINAname}{Chefe do \deptINA}
% \renewcommand{\nominataChefeINT}{Prof.~Leila Ribeiro}
% \renewcommand{\nominataChefeINTname}{Chefe do \deptINT}

% A seguir são apresentados comandos específicos para alguns
% tipos de documentos.

% Relatório de Pesquisa [rp]:
% \rp{123}             % numero do rp
% \financ{CNPq, CAPES} % orgaos financiadores

% Trabalho Individual [ti]:
% \ti{123}     % numero do TI
% \ti[II]{456} % no caso de ser o segundo TI

% Trabalho de Conclusão [tc]:
% além de definir explicitamente o nome do curso (\course) e o local
% de realização (\location), é necessário redefinir a nominata,
% pois as informações necessárias dependem do curso. Ex.:
\renewcommand{\nominata}{
       UNIVERSIDADE FEDERAL DO RIO GRANDE DO SUL\\
       Reitora: Prof\textsuperscript{a}.~Wrana Maria Panizzi\\
       Pró-Reitor de Ensino: Prof.~José Carlos Ferraz Hennemann\\
       Diretor do Instituto de Informática: Prof.~Philippe Olivier Alexandre Navaux\\
       Coordenador do curso: Prof.~Seu Creysson\\
       Bibliotecária-chefe do Instituto de Informática: Beatriz Regina Bastos Haro
}

% Monografias de Especialização [espec]:
% \espec{Redes e Sistemas Distribuídos}      % nome do curso
% \coord[Profa.~Dra.]{Weber}{Taisy da Silva} % coordenador do curso
% \dept{INA}                                 % departamento relacionado

%
% palavras-chave
% iniciar todas com letras minúsculas, exceto no caso de abreviaturas
%
\keyword{recommender systems}
\keyword{AI}
\keyword{crowdfunding}
\keyword{\LaTeX}
\keyword{ABNT}
\keyword{UFRGS}

%
% inicio do documento
%
\begin{document}

% folha de rosto
% às vezes é necessário redefinir algum comando logo antes de produzir
% a folha de rosto:
% \renewcommand{\coordname}{Coordenadora do Curso}
\maketitle

% dedicatoria
\clearpage
\begin{flushright}
\mbox{}\vfill
{\sffamily\itshape
``If I have seen farther than others,\\
it is because I stood on the shoulders of giants.''\\}
--- \textsc{Sir~Isaac Newton}
\end{flushright}

% agradecimentos
\chapter*{Agradecimentos}
Agradeço ao \LaTeX\ por não ter vírus de macro\ldots

% sumario
\tableofcontents

% lista de abreviaturas e siglas
% o parametro deve ser a abreviatura mais longa
\begin{listofabbrv}{SPMD}
        \item[AI] Artificial Intelligence
        \item[NUMA] Non-Uniform Memory Access
        \item[SIMD] Single Instruction Multiple Data
        \item[SPMD] Single Program Multiple Data
        \item[ABNT] Associação Brasileira de Normas Técnicas
\end{listofabbrv}

% idem para a lista de símbolos
%\begin{listofsymbols}{$\alpha\beta\pi\omega$}
%       \item[$\sum{\frac{a}{b}}$] Somatório do produtório
%       \item[$\alpha\beta\pi\omega$] Fator de inconstância do resultado
%\end{listofsymbols}

% lista de figuras
\listoffigures

% lista de tabelas
%\listoftables

% resumo na língua do documento
\begin{abstract}
Este documento é um exemplo de como formatar documentos para o
Instituto de Informática da UFRGS usando as classes \LaTeX\
disponibilizadas pelo UTUG\@. Ao mesmo tempo, pode servir de consulta
para comandos mais genéricos. \emph{O texto do resumo não deve
conter mais do que 500 palavras.}
\end{abstract}

% resumo na outra língua
% como parametros devem ser passados o titulo e as palavras-chave
% na outra língua, separadas por vírgulas
\begin{englishabstract}{Using \LaTeX\ to Prepare Documents at II/UFRGS}{Electronic document preparation, \LaTeX, ABNT, UFRGS}
This document is an example on how to prepare documents at II/UFRGS
using the \LaTeX\ classes provided by the UTUG\@. At the same time, it
may serve as a guide for general-purpose commands. \emph{The text in
the abstract should not contain more than 500~words.}
\end{englishabstract}

% aqui comeca o texto propriamente dito

% introducao
\chapter{Introdução}
No início dos tempos, Donald E. Knuth criou o \TeX. Algum tempo depois, Leslie Lamport criou o \LaTeX. Graças a eles, não somos obrigados a usar o Word nem o StarOffice.

\section{Figuras e tabelas}
Esta seção faz referência às Figuras~\ref{fig:ex1} e~\ref{fig:ex2}, a título de exemplo. A primeira representa o caso mais comum, onde a figura propriamente dita é importada de um arquivo \texttt{eps} ou \texttt{pdf} (aplicativos como \emph{xfig} e \emph{dia} estão entre os mais usados para gerar figuras no formato \texttt{eps}). A segunda exemplifica o uso do environment \texttt{picture}, para desenhar usando o próprio~\LaTeX.

\begin{figure}
        \centerline{\includegraphics[width=8em]{fig}}
        \caption{Exemplo de figura importada de um arquivo e também exemplo de caption muito grande que ocupa mais de uma linha na Lista de~Figuras}
        \label{fig:ex1}
\end{figure}

% o `[h]' abaixo é um parâmetro opcional que sugere que o LaTeX coloque a
% figura exatamente neste ponto do texto. Somente preocupe-se com esse tipo
% de formatação quando o texto estiver completamente pronto (uma frase a mais
% pode fazer o LaTeX mudar completamente de idéia sobre onde colocar as
% figuras e tabelas)
%\begin{figure}[h]
\begin{figure}
        \begin{center}
        \setlength{\unitlength}{.1em}
        \begin{picture}(100,100)
                \put(20,20){\circle{20}}
                \put(20,20){\small\makebox(0,0){a}}
                \put(80,80){\circle{20}}
                \put(80,80){\small\makebox(0,0){b}}
                \put(28,28){\vector(1,1){44}}
        \end{picture}
        \end{center}
        \caption{Exemplo de figura desenhada com o environment \texttt{picture}.}
        \label{fig:ex2}
\end{figure}

Tabelas são construídas com praticamente os mesmos comandos. Lembre-se, porém, que o caption das tabelas deve ir em cima.

\subsection{Classificação dos etc.}

O formato adotado pela ABNT prevê apenas três níveis (capítulo, seção e subseção). Assim, \texttt{\char'134subsubsection} não é aconselhado.

\section{Sobre as referências bibliográficas}
Recomenda-se seriamente fazer uso do pacote \emph{bibabnt}, também disponibilizado na página do UTUG~\citeyearpar{UTUG:Homepage-01}. Esse pacote provê um estilo \textsc{BibTeX} para formatação de referências bibliográficas combinando normas da ABNT e do Instituto de Informática da UFRGS\@.

As seguintes referências são colocadas aqui a título de exemplo:
\cite{Andrews:CP-91, Silberschatz:OSC-3-91, Wilson:MME-01}.

A classe \emph{iiufrgs} faz uso do pacote \emph{natbib}. Esse pacote
disponibiliza diversos comandos alternativos para
citações. Os mais úteis para nós são o \texttt{\char'134citeyearpar},
que produz somente o ano (ex.~``[\ldots] são apresentados por Baker e
Smith~\citeyearpar{Baker:PP-96}.'') e o
\texttt{\char'134citep*}, que produz a citação com a lista
completa de autores (ex.~``[\ldots] na linguagem Panda~\citep*{Assenmacher:Panda-ECOOP93}.'')

% e aqui vai a parte principal
%
% \chapter{Estado da arte}
% \chapter{Mais estado da arte}
% \chapter{A minha contribuição}
% \chapter{Prova de que a minha contribuição é válida}
% \chapter{Conclusão}
\chapter{Recommendation Algorithms}
Recommendation Algorithms are widely used in the industry today to provide useful suggestions to end-users in a completely automated manner. They are ubiquitous in modern e-commerce Web sites, where new products can be recommended based on a customer's interests and preferences, and in many other fields such as movies(Netflix) and music(Spotify). Its importance can't be overstated: the effectiveness of targeted recommendations, as measured by click-through and conversion rates, far exceed those of untargeted content.
The basic idea behind any recommender system is to obtain a utility function to estimate a user preferences towards an item.
\chapter{About Catarse}
Launched in January 2011, Catarse was the first crowdfunding platform for creative projects in Brazil. With over 7000 successfully financed projects raising R\$77 millions from 480.000 people, it's currently the largest national platform of its kind. It works similarly to most crowdfunding platforms: the project owner presents his or her idea and specifies the required investment as well as the cutoff date for the project, while offering rewards for those who back it. Projects are divided in 3 main categories: all-or-nothing, flexible and recurrent. In the first type, projects are available for backing up to 60 days and the project owner only receives the raised amount if the project's goal is met, otherwise all the money is returned to its original backers. On flexible projects the owner receives the raised amount whether the goal is reached or not. Recurrent projects are subscription based and the owner can collect the money monthly. This work will only focus on the first two types of projects.

% referencias
% aqui será usado o environment padrao `thebibliography'; porém, sugere-se
% seriamente o uso de BibTeX e do estilo abnt.bst (veja na página do
% UTUG)
%
% observe também o estilo meio estranho de alguns labels; isso é
% devido ao uso do pacote `natbib', que permite fazer citações de
% autores, ano, e diversas combinações desses
\begin{thebibliography}{este-parametro-nao-eh-usado-pelo-estilo-ABNT}

\bibitem[ANDREWS, 1991]{Andrews:CP-91} ANDREWS,
  G.~R\@. \textbf{Concurrent programming}: principles and
  practice. Redwood~City, USA: Benjamin/Cummings, 1991. 637p.
  
\bibitem[ASSENMACHER et~al.(1993)ASSENMACHER; BREITBACH; BUHLER;
  H{\"U}BSCH; SCHWARZ]{Assenmacher:Panda-ECOOP93} ASSENMACHER, H.;
  BREITBACH, T.; BUHLER, P.; H{\"U}BSCH, V.; SCHWARZ, R\@.
  Panda---supporting distributed programming in {C}++. In: EUROPEAN
  CONFERENCE ON OBJECT-ORIENTED PROGRAMMING, 7., 1993, Kaiserslautern,
  Germany. \textbf{Proceedings{\ldots}} Berlin: Springer-Verlag, 1993.
  p.361--383. (Lecture Notes in Computer Science, v.707).

\bibitem[BAKER; SMITH, 1996]{Baker:PP-96} BAKER, L.; SMITH,
  B.~J\@. \textbf{Parallel programming}. New~York: McGraw-Hill,
  1996. 381p.

\bibitem[CAROMEL; KLAUSER; VAYSSIERE, 1998]{Caromel:TSC-CPE-10-11-98}
  CAROMEL, D.; KLAUSER, W.; VAYSSIERE, J\@. Towards seamless computing
  and metacomputing in {J}ava.  \textbf{Concurrency: Practice and
  Experience}, West~Sussex, v.10, n.11--13, p.1043--1061,
  Sept./Nov.~1998.

\bibitem[FURMENTO; ROUDIER; SIEGEL, 1995]{Furmento:PDC-95} FURMENTO,
  N.; ROUDIER, Y.; SIEGEL, G\@. \textbf{Parall{\'e}lisme et
  distribution en {C}++}: une revue des langages existants. Valbonne,
  FR: I3S, Universit\'{e} de Nice Sophia-Antipolis, 1995. (RR~95-02).

\bibitem[INSTITUTE OF ELECTRICAL AND ELECTRONIC ENGINEERS,
  1995]{IEEE:Pthreads-95} INSTITUTE OF ELECTRICAL AND ELECTRONIC
  ENGINEERS\@. \textbf{Information Technology---Portable Operating
  System Interface (POSIX), Threads Extension [C Language]},
  \mbox{IEEE}~1003.1c-1995.  New~York, 1995.

\bibitem[SILBERSCHATZ; PETERSON; GALVIN, 1991]{Silberschatz:OSC-3-91}
  SILBERSCHATZ, A.; PETERSON, J.~L.; GALVIN, P.~B\@. \textbf{Operating
  system concepts}. 3.ed.  Reading, USA: Addison-Wesley, 1991. 696p.

\bibitem[UTUG(2001)UTUG]{UTUG:Homepage-01} UTUG\@. \textbf{Página do grupo
  de usuários {\TeX} da {UFRGS}}. Disponível em:
  $<$http://www.inf.ufrgs.br/utug$>$. Acesso em: maio 2001.

\bibitem[WILSON, 2001]{Wilson:MME-01} WILSON, P.~C\@. \textbf{Um
  método ótimo para o preparo de café em laboratório baseado na
  reciclagem de filtros}. 2001. 123p.  Disserta{\c{c}}{\~a}o (Mestrado
  em Ci{\^e}ncia da Computa{\c{c}}{\~a}o) --- Instituto de
  Inform{\'a}tica, Universidade Federal do Rio Grande do Sul,
  Porto~Alegre.

\end{thebibliography}

\end{document}
